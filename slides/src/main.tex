% !TEX root = main.tex

\documentclass[aspectratio=169,11pt,xcolor={dvipsnames},hyperref={pdftex,pdfpagemode=UseNone,hidelinks,pdfdisplaydoctitle=true},usepdftitle=false]{beamer}
\usepackage{../include/slide}
\usepackage{tikz}


\title{Lean Applications in the Department of Defense}
\author{Juan Carlos Cruz - ira406}
\date{ME 5703 Lean Product Development and Service Systems}

\begin{document}
    
    \begin{frame}
      \titlepage
      May 2025
    \end{frame}
    
    \begin{frame}{Introduction}
      \begin{itemize}
        \item DoD traditionally viewed as bureaucratic and slow-moving
        \item Study investigates lean/six-sigma implementation within DoD
        \item Scientometric review of 465 documents (1992-2025)
        \item Case study examination of notable works from the dataset and out of dataset example
      \end{itemize}
    \end{frame}


    \begin{frame}{Literature Search}
      \begin{itemize}
        \item In ProQuest create a search as:\\
          \begin{minipage}{\linewidth}
          \begin{spacing}{0.8}
          \noindent\texttt{AB("Department of Defense" OR "DoD" OR "DOD" OR "U.S. Army" OR} \\
          \texttt{"Department of the Army" OR "U.S. Navy" OR "Department of the Navy" OR} \\
          \texttt{"U.S. Air Force" OR "Department of the Air Force" OR "U.S. Marine Corps" OR} \\
          \texttt{"U.S. Space Force" OR "National Security Agency" OR "NSA" OR} \\
          \texttt{"Defense Intelligence Agency" OR "DIA" OR "National Geospatial-Intelligence} \\
          \texttt{Agency" OR "NGA" OR "National Reconnaissance Office" OR "NRO")} \\
          \texttt{AND FT("six sigma" OR "6$\sigma$" OR "lean six sigma" OR} \\
          \texttt{"continuous process improvement" OR "LSS") AND LA(EN)}
          \end{spacing}
          \end{minipage}
        \item 465 documents after cleaning are exported to a spreadsheet.
      \end{itemize}
    \end{frame}


    \begin{frame}{Scientometric Analysis}
      \begin{itemize}
        \item 2 Step Python program to analyze publication trends and then classify themes
        \begin{enumerate}
          \item Analyze abstracts and publication data for trends then use Latent Dirichlet Analysis on abstract to extract themes
          \item When themes extracted create a list of lean keywords related to that theme (Ex. Process Improvement: process, improvement, quality...), and classify abstracts similarity to theme
        \end{enumerate}
        \item Publication trends: Increase from 1992-2017, decline after 2017
        \item Document types: Primarily Theses and Trade journals
        \item Three themes identified: Management (54\%), Process Improvement (36\%), Continuous Learning (9\%)
      \end{itemize}
    \end{frame}



    \begin{frame}{Management and Leadership - Overview}
      \begin{itemize}
        \item Toyota's Hourensou vs. DoD's command-and-control
        \item Challenge: Implementing lean in hierarchical systems
        \item Case studies show adaptation to DoD structures
        \item Focus on communication, knowledge flow, and supplier relationships
      \end{itemize}
    \end{frame}

    \begin{frame}{Case Study: Leadership Development Strategies}
      \begin{itemize}
        \item McCants (2024): Leadership transitions in Army
        \item High turnover environment creates disruption risks
        \item Five themes for successful transitions identified
        \item Organizational knowledge mirrors \textit{genchi genbutsu}
        \item Communication emphasis parallels \textit{hourensou}
      \end{itemize}
    \end{frame}

    \begin{frame}{Case Study: Project Engineer Turnover}
      \begin{itemize}
        \item Turner (2024): Work-related stressors and turnover
        \item Strong correlation between stress and turnover intention
        \item Key stressors: workloads, deadlines, complexity, resources
        \item Directly relates to \textit{muri} (overburden)
        \item Sustainable performance requires \textit{heijunka}
      \end{itemize}
    \end{frame}

    \begin{frame}{Case Study: Contracting Environment in AFSB}
      \begin{itemize}
        \item Carlstedt (2020): Army supplier relationship management
        \item Structured contractor processes with regular reviews
        \item Standardized documentation supports lean processes
        \item Balance of accountability and partnership
        \item Reflects Toyota's \textit{keiretsu} philosophy
      \end{itemize}
    \end{frame}

    \begin{frame}{Process Improvement - Overview}
      \begin{itemize}
        \item Building culture of \textit{kaizen} (continuous improvement)
        \item Identifying and removing non-value adding steps
        \item DoD applications show significant results
        \item Case studies demonstrate adaptability to defense contexts
      \end{itemize}
    \end{frame}

    \begin{frame}{Case Study: Army's Cost-Benefit Analysis}
      \begin{itemize}
        \item Malin (2020): Analysis of Change of Command ceremonies
        \item Army lacks process to evaluate production loss costs
        \item Costs range from \$18K to \$404K per hour
        \item Recommended training-based solution and cost tools
        \item Addresses proper valuation of non-value-added activities
      \end{itemize}
    \end{frame}

    \begin{frame}{Case Study: Improving Depot Repair Lead Time}
      \begin{itemize}
        \item Richmond (2023): Lean Six Sigma in depot repair
        \item 80\% of repair lead time was non-value-added
        \item Implementation: DMAIC and process flow analysis
        \item Results: 84\% decrease in lead time (114 to 18 days)
        \item Fill rate increase from 54\% to 70\%
      \end{itemize}
    \end{frame}

    \begin{frame}{Case Study: DoD Contract Cost Overruns}
      \begin{itemize}
        \item Funches-Allen (2025): ML analysis of 524 contracts
        \item Random forest model achieved 80\% prediction accuracy
        \item Top factors: cost estimation (42\%), risk assessment (22\%)
        \item Cost overruns represent financial \textit{muda}
        \item Enables targeted improvement initiatives
      \end{itemize}
    \end{frame}

    \begin{frame}{Continuous Learning - Overview}
      \begin{itemize}
        \item Building culture through \textit{hansei} and \textit{kaizen}
        \item Digital knowledge management amplifies learning
        \item Information must be accessible at right time by right people
        \item DoD initiatives focus on data-driven learning
      \end{itemize}
    \end{frame}

    \begin{frame}{Case Study: PTSD Diagnosis Tool}
      \begin{itemize}
        \item Le (2023): Classification Automation Tool (CAT)
        \item ML ensemble methods improve diagnosis accuracy
        \item Diagnostic errors framed as waste
        \item Identified key PTSD predictors in veterans
        \item Reduces false positives and negatives
      \end{itemize}
    \end{frame}

    \begin{frame}{Case Study: Army Knowledge Management}
      \begin{itemize}
        \item VanLaar (2023): KM implementation study
        \item "People" component ranked lowest in maturity
        \item Tacit knowledge not properly shared
        \item Four knowledge transfer barriers identified
        \item Knowledge flow barriers create waste
      \end{itemize}
    \end{frame}

    \begin{frame}{Case Study: Army's Data Fabric}
      \begin{itemize}
        \item Patel (2021): Data fabric technology implementation
        \item Addresses data stovepipes and inefficient sharing
        \item Project Rainmaker enables data synchronization
        \item Eliminates digital \textit{muda}
        \item Creates pull-based data system
      \end{itemize}
    \end{frame}

    \begin{frame}{Case Study: Defense Innovation Unit}
      \begin{itemize}
        \item Established 2015 as DoD's gateway to tech companies
        \item Only DoD entity focused on commercial technology
        \item Streamlined acquisition: 60-90 days vs. years
        \item Demonstrates lean cycle time reduction
      \end{itemize}
    \end{frame}

    \begin{frame}{Case Study: GigEagle Project}
      \begin{itemize}
        \item DIU talent platform launched 2022
        \item Matches DoD with Reserve/Guard personnel
        \item AI/ML matching for short-term projects
        \item Addresses waste: underutilized talent
        \item Lean approach to human capital management
      \end{itemize}
    \end{frame}

    \begin{frame}{Conclusion}
      \begin{itemize}
        \item Lean/Six Sigma has gained traction in DoD despite hierarchical challenges
        \item Management: Adapting Toyota practices to command structures
        \item Process Improvement: Significant results (84\% lead time reduction)
        \item Continuous Learning: Addressing digital information flow
        \item Publication decline since 2017 suggests maturation or shift
        \item Lean principles successful even in bureaucratic environments
      \end{itemize}
    \end{frame}
    
    \lastslide
\end{document}